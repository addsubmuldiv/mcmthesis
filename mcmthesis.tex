%%
%% This is file `mcmthesis-demo.tex',
%% generated with the docstrip utility.
%%
%% The original source files were:
%%
%% mcmthesis.dtx  (with options: `demo')
%% 
%% -----------------------------------
%% 
%% This is a generated file.
%% 
%% Copyright 
%%     2010 -- 2015 by Zhaoli Wang
%%     2014 -- 2016 by Liam Huang
%% 
%% This work may be distributed and/or modified under the
%% conditions of the LaTeX Project Public License, either version 1.3
%% of this license or (at your option) any later version.
%% The latest version of this license is in
%%   http://www.latex-project.org/lppl.txt
%% and version 1.3 or later is part of all distributions of LaTeX
%% version 2005/12/01 or later.
%% 
%% This work has the LPPL maintenance status `maintained'.
%% 
%% The Current Maintainer of this work is Liam Huang.
%% 
\documentclass{mcmthesis}
\mcmsetup{CTeX = false,   % 使用 CTeX 套装时,设置为 true
        tcn = 86483, problem = E,
        sheet = true, titleinsheet = true, keywordsinsheet = true,
        titlepage = false, abstract = true}
\usepackage{palatino}
\usepackage{booktabs}
\usepackage{lipsum}
\usepackage{wrapfig}
\title{The \LaTeX{} Template for MCM Version \MCMversion}
\author{\small \href{http://www.latexstudio.net/}
  {\includegraphics[width=7cm]{mcmthesis-logo}}}
\date{\today}

\renewcommand\arraystretch{1.5}
\newcolumntype{I}{!{\vrule width 1.2pt}}
\newlength\savedwidth
\newcommand\whline{\noalign{\global\savedwidth\arrayrulewidth
                            \global\arrayrulewidth 3pt}%
                   \hline
                   \noalign{\global\arrayrulewidth\savedwidth}}
\newlength\savewidth
\newcommand\shline{\noalign{\global\savewidth\arrayrulewidth
                            \global\arrayrulewidth 1.5pt}%
                   \hline
                   \noalign{\global\arrayrulewidth\savewidth}}




% 以上导言区
\begin{document}
\begin{abstract}
\lipsum[1]
\begin{keywords}
keyword1; keyword2
\end{keywords}
\end{abstract}
\maketitle
\tableofcontents
% \newpage
 

% \lipsum[2]
% \begin{itemize}
% \item minimizes the discomfort to the hands, or
% \item maximizes the outgoing velocity of the ball.
% \end{itemize}
% We focus exclusively on the second definition.

% \begin{itemize}
% \item the initial velocity and rotation of the ball,
% \item the initial velocity and rotation of the bat,
% \item the relative position and orientation of the bat and ball, and
% \item the force over time that the hitter hands applies on the handle.
% \end{itemize}
% \lipsum[3]
% \begin{itemize}
% \item the angular velocity of the bat,
% \item the velocity of the ball, and
% \item the position of impact along the bat.
% \end{itemize}
% \lipsum[4]
% \emph{center of percussion} [Brody 1986], \lipsum[5]

% \begin{Theorem} \label{thm:latex}
% \LaTeX
% \end{Theorem}
% \begin{Lemma} \label{thm:tex}
% \TeX .
% \end{Lemma}
% \begin{proof}
% The proof of theorem.
% \end{proof}
\section{Introduction}
\subsection{Background}
\subsection{Restatement of the Problem}
\subsection{Our Work}


\section{General Assumptions and Data Analysis}
\subsection{General Assumptions}
\subsection{Data Analysis}
% 表格
% \begin{table}[]
%   \centering
%   \caption{My caption}
%   \label{my-label}
%   \setlength{\tabcolsep}{7mm}{
%   \begin{tabular}{l|l|l|l|l|l|l|l|l|l|l|l}
%   \shline
%   \multicolumn{2}{lI}{\textbf{asasd}} & \multicolumn{10}{l}{\textbf{ssssssssssssssssssssss}} \\ \hline
%   \multicolumn{2}{lI}{dddd}           & \multicolumn{10}{l}{ddddddddddddddddddddd}           \\ \hline
%   \multicolumn{2}{lI}{}               & \multicolumn{10}{l}{}                                \\ \hline
%   \multicolumn{2}{lI}{}               & \multicolumn{10}{l}{}                                \\ \hline
%   \multicolumn{2}{lI}{}               & \multicolumn{10}{l}{}                                \\ \hline
%   \multicolumn{2}{lI}{}               & \multicolumn{10}{l}{}                                \\ \hline
%   \multicolumn{2}{lI}{}               & \multicolumn{10}{l}{}                                \\ \hline
%   \multicolumn{2}{lI}{}               & \multicolumn{10}{l}{}                                \\ \hline
%   \multicolumn{2}{lI}{}               & \multicolumn{10}{l}{}                                \\ \shline
%   \end{tabular}}
%   \end{table}  

\section{Variable Description and Research Method}
\subsection{Variable Description}
\subsection{Research Method}
% 插图
% \begin{figure}[h]
% \small
% \centering
% \includegraphics[width=12cm]{test1.jpg}
% \caption{asssssa} \label{fig:aa}
% \end{figure}

% 引用
% \eqref{aa}
% \begin{equation}
% a^2 \label{aa}
% \end{equation}

% 公式
% \[
%   \frac{\mathrm{d} C_{1}}{\mathrm{d} t}= \left 
%   ( 1-\frac{C}{R} \right )^{P_{I}}\frac{\mathrm{d} I}
%   {\mathrm{d} t}\frac{C_{I}}{I} \eqno (3)
% \]






\section{Model Testing}

\section{Sensitivity Analysis}
\subsection{Impact of x}
\subsection{Impact of y}

\section{Strengths and Weaknesses}
\subsection{Strengths of Models}
\subsubsection{Inclusive}
\subsubsection{Quantification}
\subsubsection{Simple but Universal}
\subsubsection{Visible and Understandable}
\subsection{Weaknesses of Models}
\subsubsection{Accuracy Relies on Statistics}
\subsubsection{Xxx Is Not Included}
% 图文混排
% \begin{wrapfigure}{l}{4.5cm}%靠文字内容的左侧
%   \includegraphics[width=4cm]{test1.jpg}\\
%   \caption{Podala Palace, Tibet}\label{fig:tibet}
%   \end{wrapfigure}

To begin with, we searched a large number of papers that discuss the spread of
Ebola to help us deepen the understanding of the problem. Chowell et al. provided a
large amount of background information and their work [6] served as an important
introduction. We found that a few of the papers used the traditional epidemic model to
predict the transmission of the disease such as the SEIR model used by Althaus [11] to
estimate the reproduction number of the virus\cite{bib1,bib2} during the 2014 outbreak. Therefore, we
also applied the SEIR model in the early stage to predict the spread of Ebola. Later, we
found out that the Ebola virus has some specific feathers that also needed to be
considered and that are, the potential transmission threat posed by the highly infectious
corpses, the improved infection control and reduced transmission rate if patients can be
treated in hospitals, and the powerful intervention method: contact tracing. After taking
all these critical factors into consideration, we improved our original epidemic model.

% 论文引用
\begin{thebibliography}{99}
\bibitem{bib1} D.~E. KNUTH   The \TeX{}book  the American
Mathematical Society and Addison-Wesley
Publishing Company , 1984-1986.
\bibitem{bib2}Lamport, Leslie,  \LaTeX{}: `` A Document Preparation System '',
Addison-Wesley Publishing Company, 1986.
\bibitem{bib3}\url{http://www.latexstudio.net/}
\bibitem{bib4}\url{http://www.chinatex.org/}
\end{thebibliography}

% 附录
\begin{appendices}

\section{First appendix}

\lipsum[13]

Here are simulation programmes we used in our model as follow.\\

\textbf{\textcolor[rgb]{0.98,0.00,0.00}{Input matlab source:}}
\lstinputlisting[language=Matlab]{./code/mcmthesis-matlab1.m}

\section{Second appendix}

some more text \textcolor[rgb]{0.98,0.00,0.00}{\textbf{Input C++ source:}}
\lstinputlisting[language=C++]{./code/mcmthesis-sudoku.cpp}

\end{appendices}
\end{document}

%% 
%% This work consists of these files mcmthesis.dtx,
%%                                   figures/ and
%%                                   code/,
%% and the derived files             mcmthesis.cls,
%%                                   mcmthesis-demo.tex,
%%                                   README,
%%                                   LICENSE,
%%                                   mcmthesis.pdf and
%%                                   mcmthesis-demo.pdf.
%%
%% End of file `mcmthesis-demo.tex'.
